% !TeX root = ../star_sporting_regulations.tex
% Add the above to each chapter to make compiling the PDF easier for some editors.

\newpage

 \section{Competition Outlines}
  
\subsection{Objective of the competition}
   \begin{enumerate}
      \item \emph{What are our values? What is the ultimate goal of the competition? What do we want to achieve?}
    \end{enumerate}


    \subsection{Participants}
    \begin{enumerate}
      \item Only enrolled university (or of comparable institutes) students can participate actively in the competition.
      \item If a team member has graduated six months before the competition, they may still participate.
      \item If a team member has graduated during the eighteen (18) months before the competition and has played a significant role,
      the Team may send an application to the organiser, requesting their participation. 
      \item Students attending other universities within a reasonable distance of a participant's university (e.g. in the same city) may join
      a student team from said university.
      \item If a student's university already has a competing team, the student is not allowed to join a competing team of 
      another university.  
      \item The team size is not limited to a maximum number of members. However, the organiser may limit the allowed on-site participants to a maximum of \textcolor{red}{15} for each Team at the final event.
      \item PhD candidates still enrolled or working for the university can join the teams. 
    \end{enumerate}

    \subsection{Competition Rewards}
    \begin{center}
        \begin{tabular}{|c|c|c|} 
          \hline
          Position & Discipline & \textcolor{red}{Moneytary Award} \\
          \hline
          1st & Overall & \textcolor{red}{25\%} \\ 
          \hline
          2nd  & Overall & \textcolor{red}{15\%} \\ 
          \hline
          3rd & Overall & \textcolor{red}{10\%} \\ 
          \hline
          1st & AI racing & \textcolor{red}{15\%} \\ 
          \hline
          2nd  & AI racing & \textcolor{red}{9\%} \\ 
          \hline
          3rd & AI racing & \textcolor{red}{6\%} \\ 
          \hline
          1st & Safety & \textcolor{red}{5\%} \\ 
          \hline
          1st & Design Maturity & \textcolor{red}{5\%} \\ 
          \hline
          1st & Innovation and Sustainability & \textcolor{red}{5\%} \\ 
          \hline
          1st & Special Prize & \textcolor{red}{5\%} \\ 
          \hline
        \end{tabular}
      \end{center}
      

    \begin{enumerate}
      \item Every winner of one of the categories mentioned above will receive the following prizes:
      
      \begin{itemize}
        \item Trophy according to their position and discipline
        \item Free entry to the competition in the next year.
        \item Special prizes like tours Etc. by sponsors
        \item \textcolor{red}{Prize Money according to the table}
      \end{itemize}

      \item \textcolor{red}{No prize money is planned for the competition. Since the registration fee is intended only to incentivise the teams to compete, the fees might get split between the winning teams. \\ \\ Furthermore, Student AirRace is looking into 
      the possibility of collecting additional money as potential prize money for the teams. If the option for prize money is agreed upon, it would be split according to the percentages in the table above. }
    \end{enumerate}


    \section{Competition Processes}

    \subsection{Registration Process}
    \begin{enumerate}
      \item Teams can register for the competition on the website of Student AirRace. The sign-up page can be found at www.student-airrace.com/recruiting 
      \item Only one Team per individual university may register for the competition.  
      \item The organiser will choose the participating teams on a "First Come First Served"-Base but also
      on the perception and opinion of the organisers whether a team will be able to meet the requirements and standards of the competition. 
    \end{enumerate}

    \subsection{Later Changes to the Registration}
    \begin{enumerate}
      \item Teams may submit changes to the information they have provided in their registration up to six months before the competition enabling the teams to rebrand or change their starting numbers.
     \item If, in the organiser's opinion, a competitor is likely to fail to meet the requirements or standards
      of the competition, they are allowed to reject them after the registration process. 
      \item In case of withdrawal from the competition, the Team must notify the organisers as soon as possible. Withdrawals must be entered six months before the competition. 
      The organisers may penalise any non-existing or later withdrawals with a point deduction or ban from future events. 
    \end{enumerate}

    \subsection{Fees}
    \begin{enumerate}
    \item A participation fee of \textcolor{red}{300€} per Team will be required to participate in the competition.
    \item The fee will go into the prize money pool and be distributed to the participating teams according to the prize money table, which the organisers will determine.
    \item The fee is required as a deposit to ensure the Team's commitment to the competition and their timely arrival.
    \item The fee payment is due after \textcolor{red}{6 months} after registration and will be handled by the organisers.
    \item Teams who fail to pay the participation fee will not be allowed to participate in the competition.
    \end{enumerate}

    \subsection{Expenses}
    \begin{enumerate}
      \item The organisers will not support or reimburse the teams for travel, accommodation, or other expenses. 
      \item The competitors may search for private, university or company sponsors to cover any of their expenses.
    \end{enumerate}

    \section{Competitor Organization}

    \subsection{Just Culture}
    \begin{enumerate}
      \item The teams and the organiser commit to adopting a so-called "Just Culture" to ensure the competition's and its participants' safety.
      \item The team leaders commit themselves to learning about Just Culture and educating their team members about it.
      \item Participants who notice safety-related issues in their Team or within the competition can report them anonymously to info@student-airrace.com and are assured that it will not bear any
      negative consequences for them or their Team. 
      \item \textcolor{red}{Student AirRace also seeks to provide the teams with workshops about this topic.}
    \end{enumerate}

    \section{Practice during development}
    \begin{enumerate}
      \item During the development, and testing practice of their UAVs, the teams must meet all the laws and requirements of their flight location. 
      \item There is no maximum limit on how much time the teams invest into flying before the competition. However, a minimum flight experience on the competition UAV and for the pilots must be documented. More can be found at Section \ref{SubSec:PilotRequirements}
    \end{enumerate}

    \subsection{Team Documentation}
    The following rules are not meant to annoy the teams or make their lives hard. The competition's success is in the hands of the competitors. To grow the competition 
    sustainably, we need to keep specific aviation and safety standards. Furthermore, those rules are supposed to teach you standard aviation processes and help you learn and grow. 
    If you have any questions on approaching those flight documents, the organising Team of Student AirRace will gladly assist you. Safety and the success of this competition are in all of our hands.
    \begin{enumerate}
      \item The teams must document all flights with their prototypes in physical and digital flight books. They must note at least location, flight time, and pilot flying as well as observations, abnormal events or incidents during the flight.
      \item The teams are required to document their pilot's flights and flight time on aircrafts comparable to or the competition aircraft (e.g. Racing Quadrocopters)
      for each flight and in a cumulative manner. These documents will be checked according to \ref{SubSec:PilotRequirements} at the competition. 
      \item The teams must keep up-to-date records about the equipment installed on the UAV, including all electronic and mechanical subassemblies on the UAV. 
      Changes must be marked and recorded. 
      \item Teams must keep up-to-date records of any damages or problems on the UAV. Even if the aircraft might still be flightworthy. 
      Any repairs will have to be documented, stating the reason for the repair and a short overview of the repair process. 
    \end{enumerate}

    \subsection{Feedback for the organiser}
    \begin{enumerate}
      \item The teams are asked to inform the organisers about any obstacles or issues they encounter, which too strict or lenient regulations might cause. The organisers will gladly receive any feedback and integrate it into the regulations. 
      \item Teams must inform the organisers about incidents that lead to the UAV's loss, damage to ground structures, injury, or death. 
    \end{enumerate}



    \section{Final Competition Logistics}
    \subsection{Logistics}
    \begin{enumerate}
      \item Each Team can ship a maximum of two twenty (20) ft shipping containers to the location of the competition. The containers will be placed right next to the Team's reserved location in the paddock.
      \item The teams are responsible for shipping expenses and on-time delivery of the containers. Shipping of the containers is possible from \textcolor{red}{three months }before the final competition takes place.
      \item The organisers will not accept any other types of shipment unless additional agreements are in place beforehand.
    \end{enumerate}

    \subsection{Paddock}
    \begin{enumerate}
      \item The paddock is where the teams are allowed to work on their UAS. 
      \item Each Team gets their spot "Team Home" within the paddock (ca. 7m x 7m), where they can put up a small tent, flags Etc. Their shipped containers will also be right next to their spot. 
      \item Under no circumstances may any UAS fly within the paddock without prior approval by the organiser. Doing so will lead to heavy penalties up to disqualification. 
      \item The organiser will provide the teams with power within their team home.    
    \end{enumerate}


    \section{Final Competition Organization}

    \subsection{Scruteneering}
    \begin{enumerate}
      \item \emph{The organisers will randomly weigh vehicles.}
      \item \emph{Do bench checks and flight test checks.}
    \end{enumerate}

    \subsection{Team Briefings}
    \begin{enumerate}
      \item On each day of the competition, there will be a team briefing.
      \item There may be additional mandatory briefings for specific roles, such as pilots.
      \item The team members must attend each briefing and present at the flight site during the flights to the briefing.
      \item One point can be deducted by the organiser for each missing member during the briefings.
    \end{enumerate}

    \subsection{Parc Fermé}
    \begin{enumerate}
      \item Parc Fermé begins when the Team starts the Thrust-Stand-Event for the first time during the competition days. It ends with the completion of all three competition flights. 
      \item While Parc Fermeé is active, teams are not allowed to work on the UAVs. Only the following works are allowed to be carried out: 
      \begin{itemize}
        \item Balancing of propellers
        \item Cleaning of any parts of the aircraft
        \item Exchange of batteries for a 1:1 replacement
        \item Charging of batteries 
        \item Reading logs and taking measurements
        \item Tightening of any loose screws
        \item Adjustment of minor software parameters. 
        \item Repair of accident damage and 1:1 replacement of broken parts as long as the organiser deems the crash not to have happened on purpose. 
      \end{itemize}
      \item Work on safety-related systems may be done after submitting a written request to the organiser and getting it accepted. 
    \end{enumerate}

    \subsection{Protests}
    \begin{enumerate}
      \item Teams are allowed to protest against any decisions, rules or penalties imposed by the organiser.
      \item They must do so within 6 hours of the decision. 
      \item The protest must be sent to the organiser in written form; this can happen digitally. It must state the decision, the reason for the protest and contacts from the Team.
      \item Along with the protest, the Team will be deducted 10 Points from their final score. If the organisers agree with the protest, the teams receive those points back. 
    \end{enumerate}

    \subsection{Spare parts}
    \begin{enumerate}
      \item \emph{How many spare chassis or even spare UAVs is each Team allowed to have?}
      \item \emph{How many spare parts or parts of different specifications (e.g. different propeller pitches) is each Team allowed to have?}
    \end{enumerate}

    \section{Final Competition Disciplines}
    \subsection{Mandatory Disciplines}

    \subsection{Scoring}
    \begin{enumerate}
      \item Each Team must compete in the mandatory disciplines mentioned below. 
      \item A team can achieve a maximum of two hundred eleven (211) points which can be collected during different mandatory events.
      \item At the end of the competition and after deducting any penalties, the Team with the most cumulated points is declared the main event winner. 
      \item If two teams achieve the same amount of points, also known as a dead heat after all points have been awarded, the winner will be declared who scored more points in the efficiency discipline. 
      If both have also equally scored in this discipline, this process repeats for the safety, design maturity and, innovation \& sustainability disciplines in the hereby mentioned order. If this doesn't deliver any result, the organiser will make a decision. 
      \item The following table shows how the total points are distributed across the individual disciplines. 

      \begin{center}
        \begin{tabular}{|c|c|c|} 
          \hline
          Discipline & Points & Awardment Process \\ 
          \hline
          Hotlap Races & 75 & Time measured by organizer \\ 
          \hline
          Landing Challenge & 36 & Time measured by organizer \\ 
          \hline
          Efficiency & 25 & Time measured by organizer \\ 
          \hline
          Safety & 25 & Jury Decision \\ 
          \hline
          Design Maturity and Technical Report & 35 & Jury Decision \\ 
          \hline
          Innovation and sustainability & 15 & Jury Decision \\ 
          \hline
        \end{tabular}
      \end{center}
    \end{enumerate}

    \subsection{Comparability between runs}
    \begin{enumerate}
      \item To ensure comparability between the runs, the organisers will limit the metrological conditions during which the runs can happen. If the organiser calls the conditions fine for takeoff, the takeoff time window rule applies. \\
      The organisers will monitor the following conditions at the track during the flights:
      \begin{itemize}
        \item Precipitation
        \item Temperature 
        \item Windspeed
        \item Humidity
        \item Air Pressure
        \item GNSS Connectivity and satellite number at a fixed position on site (GPS only) 
        \item Frequency jamming
      \end{itemize}
    \end{enumerate}

    \subsection{Race Course Layouts}
    \begin{enumerate}
      \item The Race course layout will be built within a rectangular zone. This rectangle has side lengths of 1500m and 500m. The maximum height during the flight is limited to 100m Above Ground (AGL).
      \item No point within the race course layout will be further away than 1000m from the Ground Station. Also, every point of the racecourse layout will be able to be flown within the Line Of Sight of the Ground Station. 
      \item Race courses consist of an assembly of different predetermined race course elements published with the regulations.
      \item The official assembly of race course layouts for the three hot lap sessions will not be published until shortly before the competition to prevent the teams from overfitting their aircraft to one specific racecourse. 
      \item Race course assembly kits are published in Annex A. 
      \item The race courses are marked with the help of 12m high pylons or flags. 
    \end{enumerate}


    \section{Flight Operations at the final competition}
    \begin{enumerate}
      \item During the flights, the teams must assign different roles within the Team at the track, defined in the latter regulations.
      \item Each person fulfilling a role must communicate fluently to the organisers in English.
      \item A minimum of four (4) persons are required at the track to be allowed to conduct the flights.
      \item A maximum of ten (10) persons by a team are allowed at the track. For clarification: This does not affect the number of persons each Team is allowed to have in the paddock. 
      \item 
    \end{enumerate}

    \subsection{Required roles during flight}
    \begin{enumerate}
      \item \textbf{PL: Pilot}\\The pilot is responsible for flying the UAV.
      \item \textbf{AO: Airspace Observer/Spotter}\\The AO stands next to the PL and has an unobstructed view of the airspace around the drone. They talk to the PL directly in person. The PL and AO can communicate via an intercom if they stay beside each other. 
      \item \textbf{FE: Safety Engineer}\\The safety engineer is the one who was responsible for the safety report. Their task is to constantly monitor the situation and warn the others if they are getting into any problematic situations. 
      \item \textbf{FOL: Flight Operations Lead}\\The FOL makes the tactical decisions during the flight. They request Takeoff Permission from CL (Competition Lead). They manage contingency and emergency procedures. They are essentially the boss of the Team's operation.
    \end{enumerate}

    \textcolor{red}{The following seems like way too much information. I'd like to see some of those roles filled for professionalism and as a guide for the teams. However this makes the rules unnecessarily complicated.s
    \subsection{Optionsal roles during flight}
    \begin{enumerate}
      \item \textbf{FE: Flight Engineer}\\The Flight Engineer reads the data streamed by the aircraft in real-time. They monitor battery temperatures, voltages, vibration sensors Etc. and can advise the pilots.
      \item \textbf{AOGX: Additional Airspace Observers/Spotters}\\Not yet defined number of Airspace and Ground Observers according to the terrain and track layout.
      According to the terrain, placing additional observers on the field might be useful for an external look at whether the air or ground situation changes. They can be equipped with radios and talk to FOL.
      \item \textbf{PCX: Pit Crew }\\The pit crew is there to make changes on the eVTOL, which are permitted by the rules within the time limit. They also prepare the AC for the flight and ensure it is transported away afterwards. Only they are allowed to touch the aircraft during the flights. 
    \end{enumerate}

    \subsection{Requirements for the pilots}
    \begin{enumerate}
      \item Each Team must have at least three different designated pilots. The names of those pilots must be sent to the organiser at least 
      Thirty days before the competition. 
      \item Each pilot must have flown and documented at least twenty (20) hours on comparable UAVs (e.g. Racing Quadrocopters) in the past 12 months \label{SubSec:PilotRequirements}.  
      \item The pilots must communicate fluently with the organisers in English to follow safety-relevant commands during flight.
      \item The pilots may have to perform a flight test check in front of the organisers at the competition, which is just a quick test flight with any multicopter UAS to demonstrate the pilot's basic flight skills.
    \end{enumerate}

    \textcolor{red}{I'd like to cut Pit Stops from the regulations. Since we do not have an external shutdown, we shouldn'table
    put the team members under the pressure of having to swap the batteries that quickly. This could result in injuries and make everything more complicated.} 
    \subsection{Pit stops}
    \begin{enumerate}
      \item The teams may exchange batteries during the time window. 
      \item Those battery swaps are only allowed on the landing pad. 
      \item The time will continue to run down during the swaps. 
      \item Teams may only enter the flight test area after the aircraft has landed and the kill switch has been activated.  
      \item Depending on the situation, the organisers may add time to the clock to compensate for the time lost due to the suspension. 
    \end{enumerate}

    
  
    \subsubsection{Further Rules to be moved around}
    \begin{enumerate}
      \item The Team must take off vertically from the designated landing pad.  
    \end{enumerate}


    \section{Disciplines}
    \subsection{Thrust Test Stand}
    \textbf{Disciple Outline}
    The thrust test stand is a crucial part of the Scrutineering process and ensures that teams meet all the requirements outlined in the technical regulations. It also serves as a means for organisers to assess the safety and reliability of the teams' vehicles and their technical condition. This discipline requires successful performance to qualify for takeoff clearance for the Hotlap races.
    \begin{itemize}
      \item Each Team must demonstrate their vehicle's performance on the thrust test stand.
      \item Each Team will be assigned a specific time for their vehicle to be tested and must be ready at that time.
      \item The vehicle must be mounted on the thrust test stand using the adapter specified in the technical regulations.
      \item The Team must run their vehicle for at least 50\% of their battery capacity on the thrust test stand.
      \item The shape of the used thrust curve over time is at the discretion of the competitors.
      \item The Team must complete the thrust test stand event within 15 minutes. Failing to do so or not utilising at least 50\% of the battery capacity will result in a penalty.
      \item The vehicle must achieve a minimum peak force of \textcolor{red}{300N} for at least 5 seconds.
      \item The teams are responsible for securing their aircraft on the thrust test stand.
    \end{itemize}



    \subsection{Hotlap races}
    \textbf{Disciple Outline}
    \begin{itemize}
      \item There will be three (3) hot lap races on different tracks during the competition. The following rules apply to a single race.
      \item Each Team will be given a \textcolor{red}{fivteen (15) } minute time window during which they can conduct their flights. 
      \item Each Team is only allowed to set a maximum of five timed (5) laps during their time window. 
      \item The organiser will log each lap time. It counts as the time the aircraft passes the start/timing line until it has crossed the timing/finish line. 
      \item The points will be awarded in order of fastest hot lap achieved by each Team during their time window.
    \end{itemize}

    \textcolor{red}{German has had some good ideas about adding a landing challenge after the hot lap races. They are already considered within the point distribution. I'm scared to make it too complicated, though.}

    \subsubsection{Hotlap Race Rules}
    \begin{enumerate}
      \item The following rules apply in combination with the later specified Race Procedures.
      \item The lap time will be recorded according to the Hotlap Race Procedures outlined in the following Race Procedure sections.
      \item Teams are required to pass through every gate on the course. Missing a gate will result in the lap time being invalidated.
      \item If a gate is defined as a pole, teams must pass the pole on the side specified by the organisers. Failing to do so will result in the lap time being invalidated.
      \item Teams must ensure their drones remain within the flight perimeter, or geo cage, at all times. Any drone found to be flying outside the geo cage may be disqualified or face penalties.
      \item Each drone can fly their hot lap individually, with no other drones operating simultaneously to ensure safety and avoid interference during the race.
      \item The start gate can be crossed at any velocity. Teams are allowed to have a run-up, provided it does not exceed the boundaries of the location.
      \item A lap can be aborted at any time at the Team's discretion. If a lap is aborted, the time will be reset and will start again once the Team's drone crosses the start gate for a new lap.
      \item Each Team is allowed to pass through the start gate five times. However, they may conduct practice laps without crossing the start gate.  
      \item The points will be awarded in order of fastest hot lap achieved by each Team during their time window. Only the fastest lap of the competitor will be counted. Even if their second fastest lap should be faster than the one of another competitor.
      \item If two or more competitors achieve the same finish time, their points will be added together and divided equally.  
      \item Teams are expected to follow all additional rules and guidelines provided by the organisers. Dangerous flying will immediately result in disqualification. 
    \end{enumerate}


    \subsubsection{Hotlap Race Procedures}
    \textbf{Start of a Hotlap Race}
    \begin{enumerate}
      \item Each Team gets a certain time slot during which they can conduct their hot laps. 
      \item The organiser will ensure that the time is only started if a set of weather conditions is met. Afterwards, there is no going back, comparable to the process ski jumpers undergo during their competitions.
      \item The organiser calls the start of the session. Afterwards, the clock is started, and a \textcolor{red}{fivteen (15)} minute timer begins to tick down. The teams must conduct their hot laps within this time frame.
      \item The teams are free to conduct flights however they'd like to during this time as long as all other rules are followed. This is up to their strategy. One Team may fly three slow training laps and one race lap. Another team may set three race times or wait on the ground for the right wind conditions.
    \end{enumerate}
    \textbf{Suspension of a Hotlap Race}
    \begin{enumerate}[resume]
      \item The organiser can suspend the session in case of unexpected events, such as uninvolved people or aircraft entering the flight test area. A sudden change of weather which exceeds the limits set by the organiser may also lead to a suspension of the session. 
      \item In case of a suspension, the clock will be stopped. 
      \item The pilot must abort the lap immediately and slow the UAS down. Further actions will be discussed together with the organisers depending on the situation.
      \item The pilot must stay away as far as possible from uninvolved people as possible at any point in time. 
    \end{enumerate}

    \textbf{Resumption after suspension of a Hotlap Race}
    \begin{enumerate}[resume]
      \item The clock is started from where it was stopped beforehand.
      \item Depending on the situation, the organisers may add time to the clock to compensate for the time lost due to the suspension. 
    \end{enumerate}

    \textbf{Finish of a Hotlap Race}
    \begin{enumerate}[resume]
      \item The finish of a session is called after the fifteen (15) minute time window is over. After this, the teams are not allowed to start a new lap.
      \item Teams may finish the current lap in progress if it does not take longer than two minutes.
      \item Teams may prematurely communicate their wish to end the current Hotlaprace to the organiser. For example, if there's still time left but the battery charge is insufficient to conduct another safe lap.
      \item In case of a ground collision from which the Team cannot safely recover, the session is also ended. 
      \item The organiser may call an early end for a session if they deem a continuation of the flight as unsafe. The session will not be retaken.  
      \item If the Team doesn't manage to land at the landing pad, even though a safe landing is possible after the end of the race, a penalty is applied by the organisers.
    \end{enumerate}

    \subsection{AI races}
    \begin{itemize}
      \item There will be a total of three (3) hot lap races on different tracks during the competition. 
      \item Each Team will be given a fifteen (15) minute time window to conduct their flights. 
      \item Each Team is only allowed to fly a maximum of three timed (3) laps during their time window. 
      \item The organiser will log each lap time. It counts as when the start lights turn off until the aircraft has crossed the timing/finish line. 
      \item Teams can earn up to 25 Points during the competition. The exact point distribution is subject to change and will be affected by many factors, such as the number of participants.
      The teams will be notified as soon as more information is available.
      \item The points will be awarded in order of the fastest hot laps achieved by the teams during their time window.
      \item If two or more competitors achieve the same finish time, their points will be added together and divided equally.  
    \end{itemize}

    \subsection{Autonomous Races}
    The autonomous races will test the ability of the teams to develop and implement autonomous control algorithms for their drones. These races will be held on a separate track and have their own rules.

    \subsubsection{Race Rules}
    \begin{enumerate}
    \item The following rules apply in combination with the later specified Race Procedures.
    \item Each Team must design and program their autonomous control algorithms for their drones.
    \item Teams may choose to build a new drone specifically for the autonomous races as long as it meets all the technical regulations outlined in the technical regulations.
    \item Alternatively, teams may use the same drone for manual and autonomous disciplines.
    \item All calculations for autonomous control algorithms must be run onboard the drone and cannot be executed from a ground station or other external computing device.
    \item The drone may communicate with external devices to receive location data or other necessary information, but all control and decision-making processes must be performed onboard the drone.
    \item Each lap must be completed autonomously by the drone without a single intervention by a human. Otherwise, the lap will be invalidated.
    \item The drones must pass through every gate on the course. Missing a gate will result in a time penalty.
    \item If a gate is defined as a pole, drones must pass the pole on the side specified by the organisers. Failing to do so will result in a time penalty.
    \item Drones must remain within the flight perimeter, or geo cage, at all times. Any drone found to be flying outside the geo cage may be disqualified or face penalties.
    \item The start gate can be crossed at any velocity. Teams are allowed to have a run-up, provided it does not exceed the boundaries of the location.
    \item The points will be awarded in order of the fastest lap each Team achieves during the race.
    \item If two or more competitors achieve the same finish time, their points will be added together and divided equally.
    \item Teams are expected to follow all additional rules and guidelines provided by the organisers. Dangerous flying without clear intent to fly the laps properly will immediately result in disqualification.
    \end{enumerate}

    \subsubsection{Manual Control, Safety Functions, and Autonomous Flight}
    \begin{enumerate}[resume]
    \item The drone used in the autonomous race must be capable of manual and autonomous control.
    \item All drones used in the autonomous races must also be able to be manually controlled by the Team and always manually overwritable.
    \item The drone's autonomous control algorithms must have built-in safety functions such as kill switches and Return to Home (RTH) capabilities.
    \item The Team must be able to trigger these safety functions at any time during the race if necessary.
    \item The drone must be presented to the scrutineering committee before the start of the autonomous race to ensure that it meets all manual control, safety functions, autonomous flight requirements and technical regulations.
    \item The Team may use manual flight to move the drone to the designated starting position or landing spot before the race begins, interfere with the drone during the race if necessary for safety reasons, or abort a lap.
    \item Once the race begins at the start gate, the drone must autonomously fly the course without further manual flight input.
    \item Any manual flight input between the start gate and finish gate that affects the drone's flight path or speed will result in an invalid lap time.
    \item The Team may be allowed to retake the lap if the manual flight input was due to safety reasons and was not the Team's fault.
    \item Violating the rules regarding manual control, safety functions, and autonomous flight may result in disqualification.
    \end{enumerate}





    \subsubsection{Race Procedures}
    \textbf{Start of the Autonomous Race}
    \begin{enumerate}
    \item Each Team will be given a \textcolor{red}{fivteen (15)} minute time window during which they can conduct their flights. 
    \item The organiser will ensure that the time window can only start if a set of weather conditions is met. Once the race starts, there is no going back.
    \item The race will begin with the opening gate, and each drone will begin racing autonomously.
    \item The teams are free to program their drones as they see fit, as long as all other rules are followed. This is up to their strategy.
    \end{enumerate}

    \textbf{Suspension of the Autonomous Race}
    \begin{enumerate}[resume]
    \item The organiser can suspend the race in case of unexpected events, such as uninvolved people or aircraft entering the flight test area. A sudden change of weather which exceeds the limits set by the organiser may also lead to a suspension of the race.
    \item In case of a suspension, the drones must immediately stop racing, change to manual control and return to the start gate or follow instructions by the organiser.
    \item The pilot must stay away as far as possible from uninvolved people as possible at any point in time.
    \ Further action will be discussed with the organisers depending on the situation.
    \end{enumerate}

    \textbf{Resumption after Suspension of the Autonomous Race}
    \begin{enumerate}[resume]
    \item The race will resume once the suspension is lifted.
    \item Depending on the situation, the organisers may add additional time to the race or allow the competitors to recharge their batteries to a certain level to compensate for the suspension.
    \end{enumerate}



    \section{Reports}

    \subsection{Design Maturity and Technical Report}
    \subsubsection{Technical Report}
    \begin{enumerate}
    \item Each Team must submit a technical report before the competition starts outlining the technical details of their drone.
    \item The technical report must include details such as the drone's weight, dimensions, power source, flight time, range, maximum speed, and any other technical specifications required by the organisers.
    \item The technical report must also include information about the drone's control system, sensors, communication systems, and any other relevant components or subsystems.
    \item The technical report must be ten pages long, written in English, and submitted in PDF format.
    \item A jury will grade the technical report worth a maximum of 20 points.
    \item The jury will evaluate the technical report based on the level of detail provided, the quality of the engineering and design work, and the overall technical feasibility of the drone.
    \item Teams may use the technical report to highlight any innovative or unique features of their drone's design or technology.
    \item The jury may also use the technical report to determine whether a team is eligible to participate in the competition or to determine penalties or disqualifications for technical violations during the competition.
    \end{enumerate}
    
    \subsubsection{Design Maturity}
    \begin{enumerate}
    \item Each Team's drone must meet minimum standards of design maturity to be eligible for participation in the competition.
    \item The drone's physical construction must be of professional quality and meet acceptable aviation standards, which include no hanging cables, proper soldering, and proper engineering practices.
    \item The Team must demonstrate a high level of care and attention to detail in constructing and presenting their drone.
    \item The design maturity evaluation will be worth 15 points and will be based on the quality of the drone's construction and presentation.
    \item The jury may evaluate the design maturity of the drone at any point during the competition and may disqualify any team whose drone does not meet minimum design maturity standards.
    \end{enumerate}
    
    
    

    \subsection{Innovation and Sustainability Report}
    The innovation and sustainability report is an opportunity for teams to demonstrate their commitment to sustainable innovation and to showcase the potential of their drone's features for promoting sustainability in aviation.
    \begin{enumerate}
    \item Each Team must submit an innovation and sustainability report before the competition starts outlining their drone's innovative and sustainable features.
    \item The innovation and sustainability report must include details such as how the Team's drone uses sustainable materials, energy-efficient systems, or innovative features to solve a real-world problem.
    \item The innovation and sustainability report must be a maximum of 15 pages long, written in English, and submitted in PDF format.
    \item A jury will grade the innovation and sustainability report, which is worth 15 points.
    \item The jury will evaluate the innovation and sustainability report based on the level of detail provided, the originality and creativity of the Team's approach, and the potential impact of the drone's features on sustainability or innovation.
    \end{enumerate}

    \subsection{Safety Report}
    Safety is the top priority of Student AirRace. Teams must submit a safety report detailing their drone's safety features, procedures, and pilot training. The report helps teams identify and mitigate safety risks and highlight safety features. The jury evaluates the report to determine eligibility and penalties for safety violations.
    \begin{enumerate}
    \item Each Team must submit a safety report before the competition starts outlining the safety features and procedures of their drone.
    \item The safety report is a crucial aspect of the competition and is taken very seriously. Failure to submit a safety report or provide a thorough and accurate report may result in penalties or disqualification. Teams need to prioritise safety and take the safety report seriously to ensure the safety of all participants and spectators.
    \item The safety report may include details such as how the Team's drone ensures the safety of the pilot, spectators, and other aircraft in the area.
    \item The safety report must be 20 pages long at its maximum, written in English, and submitted in PDF format.
    \item A jury will grade the safety report, which is worth 25 points.
    \item The jury may evaluate the safety report based on the level of detail provided, the effectiveness of the safety features and procedures, and the overall safety of the drone and its operation.
    \item The safety report may also include details about the Team's pilot training and qualifications, as well as the training and qualifications of any personnel involved in the drone's operation or maintenance.
    \item The Team may also provide information about any redundancy features of the drone, such as redundant power sources or communication systems.
    \item The safety report may detail the total number of flight hours completed by the drone and any previous crashes or incidents, along with the lessons learned from those incidents and any changes made to improve safety.
    \item Teams may also use the safety report to highlight any safety features or procedures that may impact the drone's performance negatively but significantly increase safety.
    \item The jury will evaluate the safety report based on how well the Team can demonstrate their commitment to safety and ability to identify and mitigate potential safety risks.
    \item The jury may use the safety report to determine whether a team is eligible to participate in the competition or whether any penalties or disqualifications may be warranted in the event of a safety violation.
    \end{enumerate}
    
    

  



