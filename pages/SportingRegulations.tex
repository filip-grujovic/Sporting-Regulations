% !TeX root = ../star_sporting_regulations.tex
% Add the above to each chapter to make compiling the PDF easier in some editors.


 \section{General}
  
    Objective of the competition
    \begin{enumerate}
      \item \emph{What are our values? What is the ultimate goal of the competition? What do we want to achieve?}
    \end{enumerate}

  \section{Registration Process}
      \begin{enumerate}
        \item \emph{TO ADD: Registration Sign up process}
        \item Teams may sumbit changes to the information they have provided in their registration up to six months before the competition. This enables to teams to rebrand or change their starting numbers.
        \item Only one team per individual university may register for the competition.  
        \item There is no registration fee in order to enter the competition. 
        \item If in the opinion of the organizer, a competitor is likely to not meet the requirements or standards
        of the competition, they are allowed to reject them after the registration process. 
        \item The maximum number of attending teams might be limited by the organizers. A waiting list for the competition
        will be put in place if this is the case. 
        \item The participating teams will be chosen by the organizer. This will partly happen on a "First Come First Served"-Base but also
        on the perception and opinion of the organizers whether a team will be able to meet the requirements and standards of the competition. 
        \item In case of withdrawal from the competition, the team must notify the oganizers as soon as possible. Withdrawals must be entered until six months before the competition. 
        Any non-existing or later withdrawals may be penalized by the organizers with point deduction or ban from future events. 
      \end{enumerate}

    \section{Participants}
    \begin{enumerate}
      \item Only enrolled university (or of comparable institutes) students are allowed to actively take part in the competition.
      \item If a team member has graduated during six months before the competition, they may still participate.
      \item If a team member has graduated during the eighteen (18) months before the competition and has played a significant role within the team,
      the team may send an application to the organizer, requesting their participation. 
      \item Students attending other universities which are within a reasonable distance of a participants university (e.g. in the same city) may join
      a student team from said university.
      \item If a students university already has a competing team, the student is not allowed to join a competing team of 
      another university.  
      \item The team size is not limited to a maximum number of members. However the organizer may limit the numbers of allowed participants for each team at the final event. 
    \end{enumerate}

    \section{Student Competition}
    \begin{enumerate}
      \item \emph{This is supposed to explain the rules for a student competition. What roles do students play? 
      Are they allowed to seek professional help? Are PhD candidates allowed to join?}
    \end{enumerate}

    \section{Competition Rewards}
    \begin{enumerate}
      \item \emph{What kind of trophies will be there for the teams? Do they get any other special rights? For example free admission for the next event?}
      \item \emph{Will there be a price money? How high is it? How much for each place? How does it work from a tax point of view?  }
    \end{enumerate}




    \section{Expenses}
    \begin{enumerate}
      \item The organizers will not support or reimburse the teams for travel, accommodation or any other expenses. 
      \item The competitors may search for private, university or company sponsors to cover any of their expenses.
    \end{enumerate}

    \section{Protests}
    \begin{enumerate}
      \item Teams are allowed to launch a protest against any decisions, rules or penalties imposed by the organizer.
      \item They must do so within 6 hours of the decision. 
      \item The protest must be sent to the organizer in written form, this can happen digitally. It must state the decision, the reason for protest and contacts from the team.
      \item Along with the protest the team will be deducted 10 Points from their final score. If the organizers agree with the protest, the teams receive those points back. 
    \end{enumerate}

    \section{Liveries}
    \begin{enumerate}
      \item The Logo of the University which operates the UAV must be clearly visible from the front, left and right side. It shall not be smaller than onehundredfifty (80) mm in its smallest dimension.
      \item The competitors' number must be clearly readable from any the front. It shall not be smaller than onehundredfifty (150) mm in its smallest dimension.
      \item The teams are free to design the livery in their own style and team colors. 
      \item The teams are allowed to display sponsor logos etc. anywhere on their UAS. Those logos must not consist of any unapropriate or political messages or images. 
      \item If the organizers deem that a design element of the livery fails to meet the regulations they may cover them up or even impose penalties upon the competitor. 
    \end{enumerate}

    \section{Team Documentation}
    \begin{enumerate}
      \item The teams are required to document all flights with their prototypes in physical and digital flight books. They must note at least location, flight time, pilot flying 
      as well as observations, abnormal events or incidents during the flight.
      \item The teams are required to document their pilots personal flight time on aircrafts comparable to the competition (e.g. Racing Quadrocopters)
      for each individual flight and in a cumulative manner. 
      \item The teams must keep up to date records about the equipment installed on the UAV. This includes all electronic and mechanical subassemblies on the UAV. 
      Changes must be clearly marked and recorded. 
      \item Teams are required to keep up to date records of any damages or problems on the UAV. Even if the aircraft might still be flightworthy. 
      Any repairs will have to be documented, stating the reason of the repair and the process of the repair. 
      \item Only teams which have completed, entered and shown all the required documentation and reports are allowed to fly in the final competition. 
    \end{enumerate}

    \section{Feedback for the organizer}
    \begin{enumerate}
      \item The teams are asked to inform the organizers about any obstacles or issues they run into, which might be caused by
      too strict or too lenient regulations. The organizers will gladly receive any feedback and integrate it into the regulations. 
      \item Teams are required to inform the organizers about any incidents, which lead to loss of the UAV, damage to ground structures, injury, or death. 
      \item \emph{Do teams have to send us logs of all their flights??}
    \end{enumerate}

    \section{Just Culture}
    \begin{enumerate}
      \item The teams and the organizer commit to adopting a so called "Just Culture". This is important to ensure the safety of the competition and its participants.
      \item The team leaders commit themselves to learning about Just Culture and to educate their team members about it.
      \item Participants who notice safety related issues in their team or within the competition are able to report them anonymously  to safety@student-airrace.com. This will not bear any
      negative consequences for them nor their team. 
    \end{enumerate}

    \section{Paddock}
    \begin{enumerate}
      \item The paddock is where the teams are allowed to work on their UAS. 
      \item Each team gets their own spot "Team Home" within the paddock (ca. 7m x 7m) where they can put up a small tent, flags etc. Their shipped containers will also be right next to their spot. 
      \item Under no circumstances may a UAS fly within the paddock. This will lead to heavy penalties up to disqualification. 
      \item The organizer will provide the teams with power within their team home.    
    \end{enumerate}

    \section{Logistics}
    \begin{enumerate}
      \item Each team can ship a maximum of two twenty (20) ft shipping containers to the location of the competition. The containers will be placed right next to the teams location in the paddock.
      \item The teams are responsible for shipping expenses and on time delivery of the containers. Shipping of the containers is possible from three months before the final competition takes place.
      \item The organizers will not accept any other types of shipment unless different agreements are in place beforehand.
    \end{enumerate}

    \section{Spare parts}
    \begin{enumerate}
      \item \emph{How many spare chassis or even spare UAVs is each team allowed to have?}
      \item \emph{How many spare parts or parts of different specification (e.g. different propeller pitches) is each team allowed to have?}
    \end{enumerate}

    \section{Scoring}
    \begin{enumerate}
      \item A team can achieve a maximum of twohundred (200) points which can be collected during different events.
      \item At the end of the competition and after the deduction of any penalties, the team with the most cumulated points is declared as the winner. 
      \item If two teams achieve the exact same amount of points, also known as a dead heat, after all points have been awarded the winner will be declared by who has scored more points in the effeciency discipline. 
      If both of them have also equally scored equally in this discipline this process repeats for the safety, design maturity and innovation \& sustainability disciplines in the hereby mentioned order. If this doesn't deliver any result the organizer will make a decision. 
      \item The following table shows how the total points are distributed across the individual disciplines. 

      \begin{center}
        \begin{tabular}{|c|c|c|} 
          \hline
          Discipline & Points & Awardment Process \\ 
          \hline
          Hotlap Races & 75 & Time measured by organizer \\ 
          \hline
          Effeciency & 25 & Time measured by organizer \\ 
          \hline
          Safety & 50 & Jury Decision \\ 
          \hline
          Design Maturity & 25 & Jury Decision \\ 
          \hline
          Innovation and sustainability & 25 & Jury Decision \\ 
          \hline
        \end{tabular}
      \end{center}


    \end{enumerate}

    \section{Practice during development}
    \begin{enumerate}
      \item During development, testing practice of their UAVs the teams are required to meet all the laws and requirements of their flight location. 
      \item There is no limit on how much time the teams invest into flying before the competition. 
    \end{enumerate}

    \section{Scruteneering}
    \begin{enumerate}
      \item \emph{The organizers will randomly weigh vehicles.}
      \item \emph{Do bench checks and flight test checks.}
    \end{enumerate}

    \section{Team Briefings}
    \begin{enumerate}
      \item At each day of the competition, there will be a team briefing.
      \item For specific roles, such as pilots there may be additional mandatory briefings.
      \item Each team briefing must be attended by the team members which will be present at the flightsite during the flights to the briefing.
      \item One point can be deducted by the organizer for each member which is missing during the briefings.
    \end{enumerate}

    \section{Parc Ferme}
    \begin{enumerate}
      \item 
    \end{enumerate}

    \section{Flight Operations}
    \begin{enumerate}
      \item During the flights, the teams must assign different roles within the team at the track, which are defined in the latter regulations.
      \item Each person which fullfills a required role must be able to fluently communicate to the organizers in english.
      \item A minimum of four (4) persons are required at the track to be allowed to conduct the flights.
      \item A maximum of ten (10) persons by a team are allowed at the track. For clarification: This does not affect the amount of persons each team is allowed to have in the paddock. 
      \item 
    \end{enumerate}

    \section{Required roles during Flight}
    \begin{enumerate}
      \item \textbf{PL: Pilot}\\The pilot is the person responsible for flying the UAV.
      \item \textbf{AO: Airspace Observer/Spotter}\\The AO stands next to the PL and has an unobstructed view of the airspace around the drone. They talk to the PL directly in person. The PL and AO are allowed to communicate via intercom if they stay right next to each other. 
      \item \textbf{FE: Safety Engineer}\\The safety engineer is the one who was responsible for the safety report. Their task is to constantly monitor the situation and warn the others if they are getting into any problematic situations. 
      \item \textbf{FOL: Flight Operations Lead}\\The FOL makes the tactical decisions during the flight. They request for Takeoff Permission from CL (Competition Lead). They manage contingency and emergency procedures. They are essentially the boss of the teams operation.
    \end{enumerate}

    \section{Optional roles during Flight}
    \begin{enumerate}
      \item \textbf{FE: Flight Engineer}\\The Flight Engineer reads the data which is streamed down by the Aircraft in real time. They monitor battery temperatures, voltages, vibration sensors etc. and can give advise to the pilots.
      \item \textbf{AOGX: Additional Airspace Observers/Spotters}\\Not yet defined number of Airspace and Ground Observers according to the terrain and track layout.
      According to the terrain it might be useful to place additional oberservers on the field to have an external look at whether the air or ground situation changes. They can be equipped with radios and talk to FOL.
      \item \textbf{PCX: Pit Crew }\\The pit crew is there to make changes on the eVTOL which are permitted by the rules within the time limit. They also get the AC ready for the flight and make sure it is transported away afterward. Only they are allowed to touch the aircraft during the flights. 
    \end{enumerate}

    \section{Requirements for the pilots}
    \begin{enumerate}
      \item Each team must have at least three different designated pilots. The names of those pilots must be sent to the organizer at least 
      30 days before the competition. 
      \item Each pilot must have flown and documented at least twenty (20) hours on comparable UAVs (e.g. Racing Quadrocopters) in the past 12 months.  
      \item The pilots must be able to fluently communicate with the organizers in English, in order to be able to follow safety relevant commands during flight.
      \item The pilots may have have to perform a flight test check in front of the organizers at the competition. This is just a quick test flight with any multicopter UAS to demonstrate the pilots basic flight skills.
    \end{enumerate}


    \section{Parc Fermé}
    \begin{enumerate}
      \item Parc Fermé begins at the moment the Team starts the Thrust-Stand-Event for the first time during the competition days. It ends with the completion of all three competition flights. 
      \item While Parc Fermeé is active teams are not allowed to work on the UAVs. Only the following works are allowed to be carried out: 
      \begin{itemize}
        \item Balancing of propellers
        \item Cleaning of any parts of the aircraft
        \item Exchange of batteries for a 1:1 replacement
        \item Charging of batteries 
        \item Reading logs and taking measurements
        \item Tightening of any loose screws
        \item Adjustment of minor software parameters. 
        \item Repair of accident damage and 1:1 replacement of broken parts as long as the crash is deemed by the organizer to not have happened on purpose. 
      \end{itemize}
      \item Work on safety related systems may be done after submitting a written request to the organizer and getting it accepted by the organizer. 
    \end{enumerate}
    

    \section{Race Course Layouts}
    \begin{enumerate}
      \item The Race course layout will be build within a rectengular shaped zone. This rectangle has sidelengths of 1500m and 500m. The maximum height during the flight is limited to 100m Above Ground (AGL).
      \item No point within the race course layout is going to be further away than 1000m from the Ground Station. Also every point of the racecourse layout will be able to be flown within Line Of Sight of the Ground Station. 
      \item Race courses consist of an assembly of different predetermined race course elements which are published with the regulations.
      \item The official assembly of race course layouts for the three hotlap sessions will not be published until shortly before the competition, in order to prevent the teams from overfitting their aircraft to one specificn racecourse. 
      \item Race course assembly kits are published in Annex A. 
      \item The race courses are marked with the help of 12m high pylons or flags. 
    \end{enumerate}
  
    \section{Hotlap Sessions}
    \begin{enumerate}
      \item 
    \end{enumerate}


    \section{Start of a Hotlapsession}
    \begin{enumerate}
      \item Each team gets a certain time slot during which they are allowed to takeoff. They choose the timing of their takeoff during this time on their own. If they do not start within this time limit a DNS (Did not start) is called and the run invalid and will not be retaken. This is comparable to the process ski jumpers undergo during their competitions.
      \item The start of the session is called by the organizer. Afterfwards the clock is started and a fivteen (15) minute timer begins to tick down. The teams must conduct their hot laps within this time frame.
      \item The team must take off vertically from the designated landing pad.  
    \end{enumerate}

    \section{Suspension of a Hotlapsession}
    \begin{enumerate}
      \item The session can be suspended by the organizer in case of unexpected events such as uninvolved people or aircraft entering the flight test area. 
      \item In case of a suspension the clock will be stopped. 
      \item The pilot must abort the lap immediately and slow the UAS down. Further actions will be discussed together with the organizers depending on the situation.
      \item The pilot must stay away as far as possible from uninvolved people as possible at any point in time. 
    \end{enumerate}

    \section{Resumptions of a Hotlapsession }
    \begin{enumerate}
      \item The clock is started from where it was stopped beforehand.
      \item Depending on the situation the organizers may add additional time to the clock in order to compensate for the time lost due to the suspension. 
    \end{enumerate}

    \section{Finish of a Hotlapsession}
    \begin{enumerate}
      \item The finish of a session is called after the fivteen (15) minute time window is over. After this the teams are not allowed to start a new lap.
      \item Teams may finish the current lap in progess if it does not take longer than two minutes.
      \item In case of a ground colission from which the team is not able to safely recover the session is also ended. 
      \item The organizer may call an early end for a session if they deem a contiuation of the flight as unsafe. The session will not be retaken.  
      \item If the team doesn't manage to land at the landing pad after the end of the competition a penatly is applied by the organizers.
    \end{enumerate}

    \section{Pit stops}
    \begin{enumerate}
      \item The teams may exchange batteries during the time window. 
      \item Those battery swaps are only allowed to happen on the landing pad. 
      \item The time will continue to run down during the swaps. 
      \item Teams may only enter the flight test area after the aircraft has landed and the kill switch has been activated.  
      \item Depending on the situation the organizers may add additional time to the clock in order to compensate for the time lost due to the suspension. 
    \end{enumerate}

    \section{Comparability between runs}
    \begin{enumerate}
      \item To ensure comparability between the runs, the organizers are going to set limits on the metrological conditions during which the runs can happen. If the organizer is calling the conditions to be fine for takeoff the takeoff time window rule applies. \\
      The organizers will monitor the following conditions at the track during the flights:
      \begin{itemize}
        \item Precipitation
        \item Temperature 
        \item Windspeed
        \item Humidity
        \item Air Pressure
        \item GNSS Connectivity and satellite number at fixed position on site (GPS only) 
        \item Frequency jamming
      \end{itemize}

    \end{enumerate}

    \section{Disciplines}
    \subsection{Hotlap races}
    \begin{itemize}
      \item There will be a total of three (3) hotlap races on different tracks during the competition. 
      \item Each team will be given a fivteen (15) minute time window during which they can conduct their flights. 
      \item Each team is only allowed to fly a maximum of three timed (3) laps during their time window. 
      \item Each laptime will be logged by the organizer. It counts as the time from which the start lights turn off until the aircraft has crossed the timing/finish line. 
      \item Teams can earn up to 25 Points during the competition. The exact point distribution is subject to change and will be affected by many factors such as the number of participants.
      The teams will be notified as soon as there is more information available.
      \item The points will be awarded in order of fastest hotlaps achieved by the teams during their time window.
      \item If two or more competitors achieve the exact same finish time, their points will be added together and divided equally across the them.  
    \end{itemize}

  




