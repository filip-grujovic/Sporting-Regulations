% !TeX root = ../main.tex
% Add the above to each chapter to make compiling the PDF easier in some editors.

\chapter{Sporting Regulations}\label{chapter:Sporting Regulations}

\begin{enumerate}
  \item{Objective of the competition}
    \begin{enumerate}
      \item \emph{What are our values? What is the ultimate goal of the competition. What do we want to achieve?}
    \end{enumerate}


  \item{Registration Process}
    \begin{enumerate}
      \item \emph{TO ADD: Registration Sign up process}
      \item Only one team per individual university may register for the competition.  
      \item There is no registration fee in order to enter the competition. 
      \item If in the opinion of the organizer, a competitor is likely to not meet the requirements or standards
      of the competition, they are allowed to reject them after the registration process. 
      \item The maximum number of attending teams might be limited by the organizers. A waiting-list for the competition
      will be put in place if this is the case. 
      \item The participating teams will be chosen by the organizer. This will partly happen on a "First Come First Served"-Base but also
      on the perception and opinion of the organizers whether a team will be able to meet the requirements and standards of the competition. 
    \end{enumerate}

    \item{Participants}
    \begin{enumerate}
      \item Only enrolled university (or of comparable institutes) students are allowed to actively take part in the competition.
      \item If a team member has graduated during six months before the competition, they may still participate.
      \item If a team member has graduated during the eighteen (18) months before the competition and has played a significant role within the team,
      the team may send an application to the organizer, requesting their participation. 
      \item Students attending other universities which are within a reasonable distance to a competitors university (e.g. in the same city) may join
      a student team from said university.
      \item If a students university already has a competing team, the student is not allowed to join a competing team of 
      an other university.  
    \end{enumerate}

    \item{Student Competition}
    \begin{enumerate}
      \item \emph{This is supposed to explain the rules for a student competition. What roles do students play? 
      Are they allowed to seek professional help? Are PhD candidates allowed to join?}
    \end{enumerate}

    \item{Penalties}
    \begin{enumerate}
      \item 
    \end{enumerate}

    \item{Protests}
    \begin{enumerate}
      \item 
    \end{enumerate}

    \item{Liveries}
    \begin{enumerate}
      \item \emph{Rules for Chassis/Pilot Numbers}
    \end{enumerate}

    \item{Team Documentation}
    \begin{enumerate}
      \item The teams are required to document all flights with their prototypes in physical and digital flightbooks. They must note at least location, flighttime, pilot flying 
      as well as obeservations, abnormal events or incidents during the flight.
      \item The teams are required to document their pilots personal flight time on aircrafts comparable to the competition (e.g. Racing Quadrocopters)
      for each individual flight and in a cummulative manner. 
      \item The teams must keep up to date records about the equipement installed on the UAV. This includes all electronic and mechanical subassemblies on the UAV. 
      Changes must be clearly marked and recorded. 
      \item Teams are required to keep up to date records of any damages or problems on the UAV. Even if the aircraft might still be flightworthy. 
      Any repairs will have to be documentated, stating the reason of the repair and the process of the repair. 
    \end{enumerate}

    \item{Feedback for the organizer}
    \begin{enumerate}
      \item The teams are asked to inform the organizers about any obstacles or issues they run into, which might be caused by
      too strict or too leniant regulations. The organizers will gladly receive any feedback and integrate it into the regulations. 
      \item Teams are required to inform the organizers about any incidents, which lead to loss of the UAV, damage to ground structures, injury or death. 
      \item \emph{Do teams have to send us logs of all their flights?}
    \end{enumerate}

    \item{Just Culture}
    \begin{enumerate}
      \item \emph{The teams and the organizer commit to adjusting a so called "Just Culture". This is important to ensure the safety of the competition and its participants.}
      \item Participants who notice safety related issues in their team or within the competition are able to report them anonomously to safety@student-airrace.com . This will not bear any
      negative consequences for them nor their team. 
    \end{enumerate}

    \item{Spare parts}
    \begin{enumerate}
      \item \emph{How many spare chassis or even spare UAVs is each team allowed to have?}
      \item \emph{How many spare parts or parts of different specification (e.g. different propeller pitches) is each team allowed to have?}
    \end{enumerate}

    \item{Rating}
    \begin{enumerate}
      \item A team can achieve a maximum of twohundred (200) points which can be collected during different events.
      \item At the end of the competition and after the deduction of any penalties, the team with the most cummulated points is declared as the winner. 
      \item \emph{What happens if teams have the same amount of points?}
      \item \emph{List exact points per event.}
    \end{enumerate}

    \item{Practice during development}
    \begin{enumerate}
      \item During development, testing practice of their UAVs the teams are required to meet all the laws and requirements of their flight location. 
      \item There is no limit on how much time the teams invest into flying before the competition. 
    \end{enumerate}

    \item{Scruteneering}
    \begin{enumerate}
      \item \emph{The organizers will randomly weigh vehicles.}
    \end{enumerate}

    \item{Parc Ferme}
    \begin{enumerate}
      \item 
    \end{enumerate}

    \item{Flight Operations}
    \begin{enumerate}
      \item During the flights, the teams must assign different roles within the team at the track, which are defined in the latter regulations.
      \item Each person which fullfills a required role must be able to fluently communicate to the organizers in english.
      \item A minimum of four (4) persons are required at the track to be allowed to conduct the flights.
      \item A maximum of ten (10) persons by a team are allowed at the track. For clarification: This does not affect the amount of persons each team is allowed to have in the paddock. 
      \item 
    \end{enumerate}

    \item{Required roles during Flight}
    \begin{enumerate}
      \item \textbf{PL: Pilot}\\The pilot is the person responsible for flying the UAV.
      \item \textbf{AO: Airspace Observer/Spotter}\\The AO stands next to the PL and has an unobstructed view of the airspace around the drone. They talk to the PL directly in person. The PL and AO are allowed to communicate via intercom if they stay right next to each other. 
      \item \textbf{FE: Safety Engineer}\\The safety engineer is the one who was responsible for the safety report. Their task is to constantly monitor the situation and warn the others if they are getting into any problematic situations. 
      \item \textbf{FOL: Flight Operations Lead}\\The FOL makes the tactical decisions during the flight. They request for Takeoff Permission from CL (Competition Lead). They manage contingency and emergency procedures. They are essentially the boss of the teams operation.
    \end{enumerate}

    \item{Optional roles during Flight}
    \begin{enumerate}
      \item \textbf{FE: Flight Engineer}\\The Flight Engineer reads the data which is streamed down by the Aircraft in real time. They monitor battery temperatures, voltages, vibration sensors etc. and can give advise to the pilots.
      \item \textbf{AOGX: Additional Airspace Observers/Spotters}\\Not yet defined number of Airspace and Ground Observers according to the terrain and track layout.
      According to the terrain it might be useful to place additional oberservers on the field to have an external look at whether the air or ground situation changes. They can be equipped with radios and talk to FOL.
      \item \textbf{PCX: Pit Crew }\\The pit crew is there to make changes on the eVTOL which are permitted by the rules within the time limit. They also get the AC ready for the flight and make sure it is transported away afterward. Only they are allowed to touch the aircraft during the flights. 
    \end{enumerate}

    \item{Requirements for the pilots}
    \begin{enumerate}
      \item Each team must have at least three different designated pilots. The names of those pilots must be sent to the organizer at least 
      30 days before the competition. 
      \item Each pilot must have flown and documented at least twenty (20) hours on comparable UAVs (e.g. Racing Quadrocopters) in the past 12 months.  
      \item The pilots must be able to fluently communicate with the organizers in english, in order to be able to follow safety relevant commands during flight.
    \end{enumerate}


    \item{Parc Ferme}
    \begin{enumerate}
      \item 
    \end{enumerate}
  
    \item{Hotlap Sessions}
    \begin{enumerate}
      \item 
    \end{enumerate}

    \item{Start of a Session}
    \begin{enumerate}
      \item Each team gets a certain time slot during which they are allowed to takeoff. They choose the timing of their takeoff during this time on their own. If they do not start within this time limit a DNS (Did not start) is called and the run invalid and will not be retaken. This is comparable to the process ski jumpers undergo during their competitions. 
    \end{enumerate}

    \item{Suspension of a Session}
    \begin{enumerate}
      \item 
    \end{enumerate}

    \item{Resumptions of a Session }
    \begin{enumerate}
      \item 
    \end{enumerate}

    \item{Finish of a Session}
    \begin{enumerate}
      \item 
    \end{enumerate}

    \item{Comparability between runs}
    \begin{enumerate}
      \item To ensure comparability between the runs, the organizers are going to set limits on the metrological conditions during which the runs can happen. If the organizer is calling the conditions to be fine for takeoff the takeoff time window rule applies. \\
      The organizers will monitor the following conditions at the track during the flights:
      \begin{itemize}
        \item Precipitation
        \item Temperature 
        \item Windspeed
        \item Humidity
        \item Air Pressure
        \item GNSS Connectivity and satellite number at fixed position on site (GPS only) 
        \item Frequency jamming
      \end{itemize}

    \end{enumerate}
  
\end{enumerate}



